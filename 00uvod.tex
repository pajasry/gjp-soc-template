\chapter*{Úvod}
\addcontentsline{toc}{chapter}{Úvod} % přidá položku úvod do obsahu

Technologie v~posledních letech zažily obrovský rozmach. Téměř každý má zařízení s~nezanedbatelným výpočetním výkonem, ale málokdo ví, jak funguje a~ jak s~ním zacházet. I~přesto, že se situace postupně zlepšuje, digitální gramotnost se nezvyšuje tak, jak by bylo třeba.

Programování a~robotika se začíná prosazovat ve školách, kde většinou chybí vybavení a~výukové materiály. Běžná komerční řešení (Lego Mindstorm, Ozobot a~M-bot) jsou drahá a~většinou uživateli neposkytují dostatečnou volnost.

Tyto problémy bychom se rádi pokusili vyřešit vlastní cenově dostupnou robotickou platformou, která bude modulární, umožní programování v~grafickém prostředí i~psaním kódu. Zároveň vytvoříme vhodné výukové materiály, uživatelům dostupné přes webový portál, umožňující sebevzdělávání.

Hlavní součástí robota je DPS\footnote{Deska plošných spojů; anglicky PCB (Printed Circuit Board) je většinou sklolaminátová deska s~měděnými spoji.}, které ulehčuje uživateli vstup do světa robotiky, neboť se nemusí zabývat elektronikou, ale plně se zaměřit na programování. Vývoj desky je rozebrán v~kapitole \hyperlink{Elektronika}{Elektronika}.

Aby se uživatel mohl zaměřit opravdu jen na programování, je mu k~dispozici i~široká škála 3D tisknutelného příslušenství. Má na výběr z~několika podvozků, senzorů a~zásuvných modulů (dělo, kamera, korbu a~další). Později, pokud se rozhodne naučit se 3D modelovat, může si tvořit vlastní doplňky. O~originálních doplňcích pojednává kapitola \hyperlink{navrh}{3D návrh}.

Jedním z~požadavků bylo grafické programovací prostředí. Žádná implementace takového prostředí pro naši~platformu neexistuje, byli jsme tedy nuceni vytvořit si vlastní. S~grafickým prostředím vznikají i~knihovny pro vyšší programovací jazyky C/C++ a~Python. Veškeré materiály jsou přístupné přes náš webový portál (\href{http://niners.cz}{niners.cz}), který na sdíleném hostingu pohání framework Symfony. Softwarová část je přiblížena v~kapitole \hyperlink{software}{Software}.

Kapitola \hyperlink{vyuka}{Výuka} přibližuje způsob motivace, výuky a~tvorby materiálů, které jsou jedním z~hlavních pilířů projektu.

Čerpali jsme pouze z~ověřených a~kvalitních zdrojů, jako je dokumentace výrobců elektronických součástek, literatury a~odborných webů. Všechny obrázky pochází z~archivu autorů.